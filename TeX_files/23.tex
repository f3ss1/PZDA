Нулевая гипотеза: при проведении интервенции генеральный параметр экспериментальной группы (как правило, это средняя или распределение) не отличается от генерального параметра (метрики) в контрольной группе. Важно учитывать, что нам может быть недостаточно отсутствия изменения среднего, чтобы сказать, что нет оснований отвергать нулевую. Если нас интересует изменение распределения, нам надо еще посмотреть на дисперсию. 

Альтернативная гипотеза: Утверждение, которое противоположно нулевой гипотезе. Никогда прямо не проверяем. Делаем выбор в пользу альтернативы, если отвергаем нулевую гипотезу. 

Существуют двухсторонние альтернативы и односторонние. Различаются они тем, предполагает ли мы, как именно отличается интересующий нас параметр для экспериментальной группы по сравнению с параметром для экспериментальной группы. 

Важные понятия:

\begin{itemize}
    \item Критерий. Правило, в соответствии с которым мы отвергаем нулевую гипотезу. 
    \item p-value (p-значение) - вероятность ошибки при условии, что нулевая гипотеза верна. Допустим p-value = 5\%. Это значит, что мы будем в 5 процентах случаях считать, что метрика изменилась после интервенции, когда как на самом деле она не изменилась. 
    
    \item Значимое изменение - изменение метрики, при котором принимается альтернативная гипотеза о неравенстве средних
    \item Доверительный интервал - интервал, в пределах которого с заданной вероятностью лежит какой-то параметр. Эти интервалы могут быть использованы для проверки гипотез, мы строим вероятностные распределения параметра для двух выборок (сформированныз, например, на основании наложения или отсутствия тритмена) и затем сравниваем их средние. Если доверительные интервалы не пересекаются, то точно сможем сказать, что распределения и средние не равны. Такой метод слабо чувствителен, иногда, даже когда дов интервалы пересекаются, есть основания отвергнуть нулевую гипотезу (смотри t-test и другие метрики) 
\end{itemize}
