
\begin{enumerate}
    \item Проблема рандомизации. Сложнее в действительном мире, в программировании существуют работающие оптимизаторы.
    
    \item Если пользователей из одной группы попадает на сайт через пару дней -- что с ним делать? Информация о том, что человек был в эксперименте, хранится в его cookies. 
    
    \item Если приходит новый пользователь -- как проводить несколько экспериментов одновременно? Проблема больших компаний с кучей метрик, так как если маленькие берут выборки по 5\%, то им с головой хватает новых пользователей. Google и Yandex проводят десятки тысяч экспериментов ежегодно.
    
    \item Как можно детектировать, что разбиение плохое, на каждом этапе эксперимента? Критерий независимости. Сложнее всего в социальных сетях и агрегаторах типа Яндекс.Такси.
    
    \item Размер выборок. Есть вероятность, что фича ухудшающая. Тогда 50\% на 50\% плохая идея -- убьёт доход из-за кучи страдающих людей. С другой стороны, нам нужно достаточное количества пользователя, чтобы иметь высокую чувствительность. Если пользователей много, мы можем использовать критерии, которые хорошо работают с теми или иными распределениями. Если очень много -- с нормальным распределением (ЦПТ). Берем размер выборки, исходя из доверительного интервала, который нам интересен. 
    
\end{enumerate}
