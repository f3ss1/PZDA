\begin{enumerate}
    \item Word Error Rate (WER) -- доля ошибочных слов. 
    
    $WER = \frac{S+D+I}{N}=\frac{S+D+I}{S+D+C}$, где 
    
    \colorbox{Apricot}{S} -- кол-во замен, \colorbox{Aquamarine}{D} -- удалений, \colorbox{Salmon}{I} -- вставок, C -- совпадений.
    
    В примере, где текст "Quick brown fox jumped over a lazy dog" 
    
    был распознан как "Quick \colorbox{Apricot}{brow} \colorbox{Aquamarine}{an} fox jumped over \colorbox{Salmon}{\textcolor{Salmon}{a}} lazy dog" 
    
    $WER = \frac{1+1+1}{8}=\frac{3}{8}$. Процент неправильного распознавания слов речи – это только количественный показатель точности распознавания (количество ошибок распознавания на фразу или слово), но не вероятность распознавания (вероятность неправильного распознавания слова во фразе), потому как он не ограничивается интервалом вероятности $[0; 1]$ и не имеет верхнего предела. Например, представим, что диктор произнес фразу, состоящую из 10 слов, но система ее полностью распознала неправильно и предложила гипотезу из 12 других слов. В этом случае, $WER=120 \%$ $(S=10, I=2, H=D=0)$, и это означает, что показатель точности отрицательный $(-20\%)$, что не имеет смысла с точки зрения теории вероятностей.
    \item CER -- то же самое, но посимвольно. Пробелы считаются. Возможна стратификация или придание весов по типу датасета или тематике.
    \item В Attend \& Spell можно просто посчитать Евклидово расстояние между векторами.
    \item В LAS минимизируем кросс-энтропию.
\end{enumerate}
