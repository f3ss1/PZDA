	\section{Как обычно устроена рекомендательная система? Опишите шаги отбора кандидатов, ранжирования, переранжирования. Приведите примеры, как каждый из них может быть устроен
} 



В рекомендательной системе может участвовать очень большое количество то- варов. При каждом посещении пользователем веб-страницы, где есть блок рекомендаций, необходимо выдать ему k наиболее подходящих товаров, причём достаточно быстро (пользователь не может ждать минуту, пока загрузится страница). В хоро- шей рекомендательной системе участвуют сотни признаков — их вычисление для каждого товара, а затем ещё и применение ко всем товарам градиентного бустинга или графа вычислений вряд ли получится успеть сделать за 1 секунду. Из-за этого рекомендательные системы работают в несколько этапов: обычно всё начинается с отбора кандидатов, где быстрая модель выбирает небольшое количество (тысячи или десятки тысяч) товаров, а затем только для этих товаров вычисляется полный на- бор признаков и применяется полноценная модель. В качестве быстрой модели может выступать линейная модель на нескольких самых важных признаках или, например, простая коллаборативная модель.

\subsection{Отбор кандидатов}

Отбор кандидатов это про то, как выбрать подмножество предметов, которые пользователю интересны. 

\begin{itemize}
    \item те же предметы, производители, что и в истории пользователя 
    
    \item самое популярное сейчас
    
    \item Матричное разложение: пользователь - исполнитель. $r_{user, singer} \approx <p_{user}, p_{singer} >$
    
    \item на основе сходства: $p_{user}, q_{item}$  -  эмбеддинги (например, из LFM). Ищем K items с максимальным значением скалярного произведения $<p_{user}, q_{item}>$. Ищу вещи, которые ближе всего к пользователю. 
    
    
\end{itemize}


О ранжировании  смотрите здесь
\url{https://developers.google.com/machine-learning/recommendation/dnn/re-ranking}

и здесь \url{https://github.com/hse-ds/iad-applied-ds/blob/master/2020/lectures/lecture03-recommender.pdf}

После отбора кандидатов следует ранжирование

Переранжирование - помимо максимизации вероятности клика, мы хотим учесть бизнес-требования. Например, мы не хотим показать видео от одного и того же автора. Мы пересортировываем наш изначальное ранжирование. Так мы учитываем дополнительные требования к нашей рекомендациии. 
