Нужна для учета неявной информации.

Бывают ситуации, когда рейтинги специфичные. Например, для пар $u, i$ мы либо знаем, что взаимодействие было, либо не знаем ничего. Например, мне отгрузили информацию про то, как пользователи ходят по интернету. Я знаю, что он там только был. Не знаю, понравилось ли ему. Или человек как овощ смотрел Netflix -- и что с ним делать? Целевая переменная принимает одно значение, и другие модели can't help. Есть iALS -- implicit alternating least squares. 

Ввожу показатель $S_{u,i} = 1$ (взаимодействие было) или 0 (не было). И показатель $C_{u, i}=\alpha$ , если $S_{u,i}=1$, и 1 в обратном случае. $\alpha \sim$ 50

Изначально все i, с которыми пользователь взаимодействовал -- положительный пример. Нет -- отрицательный. Но это неправильно. Вдруг они понравятся пользователю, когда увидит? Поэтому мы просто учитываем отрицательную информацию с очень маленьким весом. За это отвечает $C_{u, i}$. Так я создаю себе второй класс i, с которыми пользователь не взаимодействовал. 

Суммируем $C_{u, i}$ по всем айтемам, аналогично LFM:

\begin{equation}
    \sum^{n}_{u=1}\sum^{m}_{i=1}C_{u, i}(S_{u, i} - b_i - b_u - < p_u, q_i > )^2 \rightarrow \min b_u, b_i, p_u, q_i 
    \label{eq:18_theor}
\end{equation}
