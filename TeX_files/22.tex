\section{В чем отличия прямого и обратного эксперимента?}

Прямой эксперимент - проверяют эффект от внедрения. То есть мы проводим какое-то изменение в организации продукта, системы и т.Д. После указанного измерения мы, убедившись, что наложение тритмента произошло случайно, сравниваем две группы (контрольную и экспериментальную)  и делаем вывод о наличии различий по интересующим нас параметрам. 

Обратный эксперимент - мы рассматриваем в качестве базовой ту группу, что находится в условиях, которые мы раньше считали новыми. Теперь интервенция заключается в том, чтобы поместить испытуемых в дефолтные условия, то есть те которые выступали контрольными в прямом эксперименте. 

Зачем нам может это понадобиться:

"You can also do what is called a reverse A/B test where you make the new experience the default, and make a test group the old site. This can sometimes be useful for code development reasons – for example, to implement a new feature may have required significant code rework, so the reverse test is not exactly the old site but close enough to ensure that the new changes are an improvement. It’s a form of back testing after a change to make sure your final implementation got the test results you were expecting." 

\url{https://alankent.me/2019/01/19/the-value-of-a-b-testing/}
