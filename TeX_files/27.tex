Помимо очевидных проблем вроде влияния на это качество сезонности (например, мы посмотрели данные только в будние дни, а в выходные все может сильно испортиться), есть еще одна менее очевидная проблема, связанная с множественной проверкой гипотез. 

Если мы часто (например, раз в час) проверяем нашу метрику, то это равносильно тому, что у нас есть условная тысяча метрик и мы проверяем, не ''прокрасилась'' ли хотя бы одна из них. Здесь надо вспомнить про \mbox{p-value}. Пусть оно у нас равно, например, 1\%. Это значит, что через 14 дней метрика может оказаться и сверху, но с вероятностью 1\% значимое изменение при этом не произошло. Если мы меряем часто (например, 100 раз за время эксперимента), то получается, что вероятность ошибки уже не 1\%, а $(1 - 0.99^{100})$, потому что все вероятности ошибки друг на друга накладываются и хотя бы один раз из ста мы теперь вполне можем получить значимое изменение ошибочно просто потому что так работает случайность.

Если мы хотим так часто тренькать значимость, то нам надо сильно занижать p-value, но лучше так просто вообще не делать, а спокойно дождаться конца эксперимента.