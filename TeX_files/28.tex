Пример с лекции:

Предположим, что мы хотим повышать выручку магазина. Тогда главной, у нас, что логично, является выручка. Эта метрика шумная, поэтому думаем дальше. 

За выручкой может пойти, например средний чек / число купивших пользователей. На средний чек влиять проще, т.к. эта метрика более чувствительная. Но средний чек это покупки со всего магазина (а не только через разрабатываемый нами элемент).

Мы можем считать, что средний чек зависит от выручки проданных через наш элемент товаров. Но это утверждение нетривиальное: например, про связь выручки и среднего чека (по логике) мы можем сказать, что выручка есть средний чек, умноженный на число пользователей, а здесь связь уже не такая прозрачная и ее надо проверять.

Выручка проданных товаров через наш элемент связана с CTR-ом элемента (показателем, насколько часто элемент кликают)

Теперь можно перейти в метрики, которые с как таковым AB-тестированием не связаны, в оффлайн-метрики ранжирования: наверное, CTR вырастет, если мы будем лучше ранжировать по какой-то логике ранжирования.

Ну а тут уже можно в целом опуститься и до ROC AUC на валидационной выборке.

Итого иерархия:

\begin{itemize}
	\item Выручка
	
	\item Средний чек / Число купивших пользователей
	
	\item Выручка проданных товаров через наш элемент
	
	\item CTR элемента
	
	\item Оффлайн метрики ранжирования
	
	\item ROC AUC на валидационной выборке
\end{itemize}

Влиять мы хотим на первую (выручка), проще всего проверить последнюю. Тут теперь можно подрубать обучение и проверять, что иерархия верная и что рост последней в конечном счете приводит к росту выручки.