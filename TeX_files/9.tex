Предпосылка: и $x$, и $z$ имеют размерность $D$.

\subsection{Функция преобразования}

\begin{equation}
	z = f(x) = 
		\begin{cases}
			&z_{1:d} = x_{1:d}\\
			&z_{d+1:D} = x_{d+1:D} \odot \exp[s(x_{1:d})] + t(x_{1:d}) 
		\end{cases}
\end{equation}

Здесь:

\begin{itemize}
	\item $z_{1:d}$ --- первые $d$ компонент вектора $z$;
	
	\item $s(x_{1:d})$ и $t(x_{1:d})$ --- нейронные сети с $d$ входами и $(D-d)$ выходами;
	
	\item $\odot$ --- поэлементное умножение.
\end{itemize}

\subsection{Якобиан}

Матрица первых производных нижнетреугольная и выглядит как-то так:

\begin{equation}
	\frac{\partial f(x)}{\partial x} = 
	\begin{pmatrix}
		\mathbb{I}_d & 0 \\
		\dfrac{\partial z_{1:d}}{\partial x_{1:d}}& \text{diag}\{\exp[s(x_{1:d})]\}
	\end{pmatrix}	
\end{equation}

Но нас интересует на сама матрица, а ее определитель, а он для нижнетреугольной матрицы записывается просто как:

\begin{equation}
	\left|\det \frac{\partial f(x)}{\partial x}\right| = \exp\left[\sum\limits_{j = d+1}^{d} s(x_{1:d})\right]
\end{equation}

\subsection{Обратная функция}

Тут тривиально:

\begin{equation}
	x = f^{-1}(z) = 
	\begin{cases}
		&x_{1:d} = z_{1:d}\\
		&x_{d+1:D} = [z_{d+1:D} - t(z_{1:d})] \odot \exp[-s(z_{1:d})]  
	\end{cases}
\end{equation}