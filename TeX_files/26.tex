\textbf{P-value} --- некоторая вероятность ошибки при условии, что нулевая гипотеза при этом верна.

P-значение обычно либо задается в начале эксперимента, либо вообще диктуется какими-то нормами компании. Фактически это некоторый показатель того, как часто мы будем ошибаться при интерпретации результатов: если p-значение равно 5\%, то это значит, что в 5\% серых экспериментов (в которых метрика на самом деле не изменилась) мы будем считать, что на самом деле метрика не изменилась. Во всяких крупных компаниях типа Яндекса и Гугла это значение имеет порядок десятых или сотых процента. 